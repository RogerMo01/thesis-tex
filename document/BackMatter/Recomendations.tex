\begin{recomendations}
    En base a los resultados obtenidos durante la experimentación, se proponen las siguientes recomendaciones para mejorar el desempeño, 
    la escalabilidad y la experiencia de usuario del sistema:
    \begin{itemize}
        \item \textbf{Optimización del selector de planta en el módulo administrativo:} 
        Se recomienda mejorar la velocidad de carga del selector de plantas utilizado en la gestión de la sección de 
        agrupación por aplicaciones. Actualmente, la carga de una gran cantidad de registros provoca 
        latencias que afectan la experiencia del usuario. Para mitigar este problema, se sugiere implementar 
        técnicas de carga diferida (lazy loading), optimizando así el tiempo de respuesta y la usabilidad 
        del módulo administrativo.
        \item \textbf{Mejora del ranking:} 
        Es aconsejable refinar el algoritmo de ranking para la búsqueda contextual, de modo que se prioricen los resultados 
        más relevantes. Asimismo, se sugiere implementar un sistema de sugerencias de búsqueda que ayude a mitigar errores 
        tipográficos, mejorando así la precisión y usabilidad del sistema de búsqueda.
        \item \textbf{Optimización de operaciones del módulo administrativo:} 
        Se identificó que el recómputo de los vectores tras las operaciones de creación, edición o eliminación de registros 
        provoca una demora temporal en el funcionamiento del sistema. Por ello, se recomienda optimizar este proceso, ya 
        sea mediante la implementación de técnicas de actualización incremental o mediante la gestión asíncrona de estas 
        operaciones, para minimizar la latencia y garantizar una experiencia fluida para el usuario administrador.
        \item \textbf{Implementación de un rol de colaborador:} 
        Actualmente, el acceso al módulo administrativo está restringido únicamente a un usuario con permisos de administración completa. 
        Para permitir la contribución de otros usuarios sin comprometer la integridad del sistema, se recomienda 
        la creación de un rol de colaborador, el cual tendría permisos limitados para gestionar y enriquecer la base 
        de datos sin acceder a funciones críticas. Este nuevo rol permitiría acciones como la adición de nuevas monografías 
        o la propuesta de modificaciones en los registros existentes, sujetas quizás, a aprobación por parte de un administrador. 
        La implementación de este sistema contribuiría al crecimiento del sitio y facilitaría su mantenimiento a largo plazo.
    \end{itemize}
    La implementación de estas recomendaciones permitirá mejorar significativamente el rendimiento y la usabilidad del sistema, 
    garantizando una mejor experiencia para los usuarios finales y administradores. Además, optimizará la eficiencia de las búsquedas 
    y agilizará la gestión de datos, lo que contribuirá a la escalabilidad y sostenibilidad del proyecto a largo plazo. Al adoptar 
    estas mejoras, se reforzará la estabilidad del sistema, asegurando su adecuado funcionamiento en un entorno de producción y 
    facilitando su evolución futura en función de las necesidades de los usuarios.
\end{recomendations}
