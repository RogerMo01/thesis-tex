\begin{conclusions}
    Este trabajo culmina con el desarrollo exitoso de BotaniQ, una plataforma web 
    que aborda el problema científico de la necesidad de una solución computacional 
    para acceder, manipular y organizar eficientemente la información contenida 
    en la obra de Juan Tomás Roig y Mesa, "Plantas medicinales, aromáticas o venenosas de Cuba". 
    El objetivo principal, que comprendía la implementación de técnicas de NLP para 
    la extracción y estructuración de la información, junto con el desarrollo de 
    un sistema web funcional para su gestión y consulta, se ha logrado satisfactoriamente.

    La hipótesis que guiaba esta investigación, la cual afirmaba que la combinación 
    de técnicas de NLP, con un sistema web basado en un modelo relacional, permitiría 
    extraer y estructurar la información de la obra de Roig para crear un sistema 
    de gestión eficiente y accesible, se ha validado con el desarrollo de BotaniQ. 
    La plataforma resultante no solo digitaliza la información del libro, sino que 
    la organiza de manera que facilita su consulta a través de una interfaz intuitiva 
    y un potente sistema de búsqueda contextual, basado en el Modelo de Espacio Vectorial. 
    El uso de PostgreSQL, aprovechando su capacidad para almacenar datos JSON, 
    permitió mantener la flexibilidad necesaria para representar las monografías 
    con sus diferentes secciones, a la vez que se garantizaba la integridad y 
    consistencia de una base de datos relacional.

    BotaniQ cumple con los requerimientos del Jardín Botánico Nacional de Cuba, 
    ofreciendo una herramienta moderna y eficiente para la difusión y conservación del 
    conocimiento sobre la flora medicinal cubana. El sistema permite acceder a información 
    detallada sobre las plantas, incluyendo sus nombres científicos, descripciones botánicas, 
    propiedades medicinales y aplicaciones. Además, el módulo administrativo facilita 
    la gestión y actualización de la información por parte de usuarios autorizados, 
    garantizando la calidad y confiabilidad de los datos.

    Los experimentos realizados confirman la eficiencia del sistema tanto en la 
    presentación de la información como en el mecanismo de búsqueda. Sin embargo, 
    se identificaron oportunidades de mejoras que, aunque necesarias para un 
    funcionamiento óptimo en un entorno de producción, no demeritan el logro principal 
    de este trabajo: la creación de una plataforma funcional y viable que demuestra 
    la eficacia del NLP y los sistemas web para la gestión de información científica 
    en el ámbito de la botánica. Este proyecto sirve como punto de partida para 
    futuras investigaciones y el desarrollo de soluciones similares aplicables a 
    otros recursos bibliográficos del campo.
\end{conclusions}
