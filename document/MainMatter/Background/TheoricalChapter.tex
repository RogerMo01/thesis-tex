\chapter{Marco te\'orico-conceptual}\label{chater: teoricalChapter}

La creciente necesidad de preservar y sistematizar el conocimiento sobre la flora cubana 
requiere  del uso de tecnologías computacionales para facilitar su organización, 
acceso y análisis. La digitalización de datos no solo mejora la estructuración 
de la información, sino también su consulta y utilización en la investigación científica 
y la práctica médica. En este contexto, el Procesamiento de Lenguaje Natural (NLP, por sus siglas en inglés) 
y las bases de datos se presentan como herramientas clave para gestionar de manera eficiente 
y dinámica este conocimiento. El presente capítulo explora los fundamentos teóricos 
y metodológicos que respaldan el desarrollo de una solución computacional destinada a integrar 
y gestionar dicha información.

\section{Plantas: una visión general desde la medicina tradicional}
Las plantas son la base de la vida en la Tierra. Son los principales productores de oxígeno 
y alimento para la mayoría de los ecosistemas, jugando un papel fundamental en el equilibrio 
de nuestro planeta. Además, desde tiempos antiguos, las plantas han sido mucho más que 
alimento: han representado una fuente inagotable de remedios naturales, esenciales para la 
salud y el bienestar humano.

La flora de Cuba es un tesoro de la naturaleza, rica en diversidad y con alto nivel de 
endemismo. La ubicación geográfica de la isla ha generado una evolución única de las plantas, 
creando un gran número de especies que no se encuentran en ninguna otra parte del mundo. 
La isla ha actuado como un laboratorio natural, donde la flora ha podido desarrollarse a lo 
largo de millones de años, dando lugar a una gran variedad de formas y colores. Algunas de 
estas plantas están adaptadas a las condiciones específicas de cada región de la isla, desde 
las zonas costeras hasta las montañas. La flora cubana es un ejemplo fascinante de la 
capacidad de la naturaleza para generar vida en entornos únicos e invita a la exploración y 
la conservación de este patrimonio natural \cite{Pocs1988}.

Las plantas medicinales han sido, y continúan siendo, una fuente invaluable de compuestos 
químicos con aplicaciones diversas en la salud humana. Desde aliviar síntomas menores hasta 
tratar enfermedades complejas, sus usos abarcan un amplio espectro terapéutico, 
incluyendo analgésicos, antiinflamatorios, antibióticos y tratamientos para afecciones 
cardiovasculares, digestivas y respiratorias.

Según el Dr. Francisco J. Morón Rodríguez en su artículo 
\textit{"Necesidad de investigaciones sobre plantas medicinales"} \cite{Moron2007}, 
el botánico norteamericano James A. Duke estima que menos del 1\% de las más de 90 mil especies 
de plantas de bosques de América Latina han sido investigadas químicamente. Además, el autor expresa:

\begin{quote}
    \textit{``Las cifras del Dr. Duke, nos hacen reflexionar  en que
    apenas conocemos las potencialidades terapéuticas de las plantas
    medicinales, el clásico símil del iceberg, para expresar la relación entre
    lo que conocemos o vemos que es mucho menor que lo oculto o desconocido,
    resulta insuficiente, porque esos témpanos de hielo flotando a la deriva
    muestran aproximadamente un cuarto de su masa total.''}
\end{quote}

El autor, también subraya el prólogo del libro \textit{``Plantas medicinales, aromáticas o venenosas de Cuba''} \cite{Roig1945}
de Juan Tomás Roig y Mesa, donde hace un llamado a la comunidad científica a comprobar, 
mediante investigaciones multidisciplinarias, los efectos de las plantas medicinales tradicionales.

Si bien la investigación científica continúa explorando y validando sus propiedades, 
la tradición ancestral en el uso de plantas medicinales ofrece un rico acervo de conocimiento 
para el desarrollo de nuevos fármacos y terapias.

La obra \textit{``Plantas medicinales, aromáticas o venenosas de Cuba''} \cite{Roig1945} de Juan Tomás Roig y Mesa 
representa un hito fundamental en el estudio de la flora medicinal cubana. 
Su exhaustiva compilación de información de las plantas, junto con descripciones botánicas 
detalladas y usos tradicionales, constituye una base inestimable para investigaciones 
posteriores. La obra de Roig no solo documentó un vasto conocimiento popular sobre las plantas 
medicinales cubanas, sino que también sentó las bases para la investigación científica 
rigurosa en este campo, dejando un legado invaluable para la fitoterapia y la conservación 
del patrimonio botánico de la isla.

Como parte del prólogo a la primera edición de la obra, Roig hace un llamado a los hombres 
de ciencia para que emprendan el estudio metódico de la flora médica y toxicológica cubana. 
Además, resalta la utilidad de algunas secciones pensando en una posible cultivación a escala 
comercial para la exportación.

A pesar de los avances científicos logrados en más de 60 años desde la primera publicación 
de la obra de Roig, la afirmación del Dr. en Ciencias Biológicas Víctor R. Fuentes Fiallo 
-- \textit{``el viejo sueño del doctor Juan Tomás Roig sigue siendo eso: un sueño''} -- \cite{Fiallo2009} 
pone de manifiesto que la ambiciosa visión que tenía Roig, aún no se ha materializado plenamente.


\subsection{Nomenclatura y clasificación}
Todas las especies de seres vivos conocidas por la humanidad se nombran según un 
sistema científico que regula la nomenclatura biológica. Este sistema estandarizado, 
establecido por organismos internacionales, busca asegurar que cada especie tenga 
un nombre único y universalmente aceptado, lo que facilita su identificación y 
clasificación dentro de la comunidad científica. En el caso de las plantas, 
la nomenclatura científica está regulada por el Código Internacional de Nomenclatura 
para Algas, Hongos y Plantas \cite{Mcneil2012}. Cada nombre científico debe estar en latín 
y consta de tres partes fundamentales: un nombre genérico que identifica el \textit{género}, 
un epíteto específico que distingue a la \textit{especie} dentro del género y el 
nombre del autor o \textit{autores} que describe oficialmente la especie.

Además de las tres categorías anteriores, algunos nombres científicos pueden incluir 
otras categorías para definir subgrupos dentro de una especie. Cuando una especie 
tiene diferencias geográficas, morfológicas o ecológicas significativas pero aún 
pertenece a la misma especie, se clasifica en \textit{subespecies}. 
La \textit{variedad} es una categoría que agrupa individuos con variaciones de 
carácter local que pueden aparecer dentro de una misma población. 
La \textit{forma} es una categoría taxonómica que representa una 
modificación ocasional de la especie, asociada o no a la distribución geográfica.
La \textit{familia} es un rango de clasificación taxonómica, que constituye un conjuntos 
de géneros entre los que se reconocen varios caracteres comunes importantes, 
y una \textit{subfamilia} es una subdivisión dentro de una familia. \cite{Romero2017}

Las plantas a menudo reciben diferentes nombres vulgares según la región y la 
cultura, reflejo de observaciones locales sobre sus propiedades, apariencia o 
historia. Esta diversidad de nombres, transmitidos oralmente, enriquece el 
conocimiento tradicional, pero puede complicar su identificación científica.



\section{Procesamiento del Lenguaje Natural: fundamentos y aplicaciones}\label{section: nlp}
La complejidad y diversidad del lenguaje humano nos diferencia del resto de las 
especies. Nuestra capacidad de comunicarnos a través del lenguaje ha sido 
fundamental para el desarrollo de la civilización, permitiendo la transmisión de 
conocimiento cultural, científico y tecnológico.

El Procesamiento del Lenguaje Natural (NLP, por sus siglas en inglés) es un campo 
de la ciencia de la computación que busca dotar a las computadoras 
de la capacidad de entender, interpretar y generar lenguaje humano. 
Esto es importante porque nos permite que las computadoras puedan comunicarse con 
nosotros de forma natural, aprender de la inmensa cantidad de información escrita 
en nuestro idioma, y profundizar nuestra comprensión científica de cómo funciona 
el lenguaje \cite{Russell2020}.

Como parte del desarrollo de esta rama de la ciencia de la computación, y anterior 
al auge de los últimos años de los randes modelos de lenguaje (LLM, por sus siglas en inglés), 
se identificaron tareas comunes que buscan permitir a las computadoras entender, 
interpretar y generar lenguaje humano. Siguiendo la convención establecida en 
el libro \textit{``Artificial intelligence: A modern approach''} de Russell y Norvig \cite{Russell2020}, 
se conservará la terminología original en inglés para describir cada una de ellas.

\begin{itemize}
    \item \textbf{Speech recognition}: El reconocimiento de voz consiste en convertir 
    el habla humana en texto escrito. Los sistemas modernos tienen una tasa de error 
    bastante baja (entre un 3\% y 5\%), comparable a la de un transcriptor humano.
    \item \textbf{Text-to-speech}: Es el proceso inverso al reconocimiento de voz: 
    transformar texto escrito en habla. El objetivo es que la voz generada suene natural, 
    con pausas y énfasis apropiados. Se está avanzando en la creación de voces con 
    diferentes acentos e incluso imitando voces de celebridades.
    \item \textbf{Machine translation}: La traducción automática implica la traducción 
    de texto de un idioma a otro. Los sistemas de traducción aprenden a partir de grandes 
    conjuntos de textos en dos idiomas (corpus bilingües) y se enfocan en traducir no 
    solo palabras individuales, sino también el significado y la estructura gramatical 
    de las oraciones.
    \item \textbf{Information extraction}: Consiste en extraer información específica 
    de un texto. Por ejemplo, se puede utilizar para resumir textos, extraer direcciones 
    de páginas web o información meteorológica de informes o datos de tablas. 
    La dificultad de la tarea depende de la estructura del texto; un texto bien 
    estructurado es más fácil de procesar que un texto no estructurado.
    \item \textbf{Information retrieval}: La recuperación de información se enfoca 
    en encontrar documentos relevantes para una consulta dada. Los motores de búsqueda 
    de internet son un ejemplo claro de sistemas que realizan esta tarea a gran escala. 
    El objetivo es devolver los documentos más pertinentes a la búsqueda del usuario.
    \item \textbf{Question answering}: A diferencia de la recuperación de información, 
    esta tarea busca responder preguntas específicas en lugar de simplemente mostrar 
    una lista de documentos. Los sistemas modernos utilizan técnicas complejas para 
    comprender el significado de la pregunta y encontrar la respuesta correcta en 
    una base de datos de información o en Internet.
\end{itemize}

En años recientes, los LLM han revolucionado el campo del NLP al demostrar capacidades 
sobresalientes en tareas como la respuesta a preguntas, la traducción automática y 
la generación de texto.

El artículo \textit{``Large Language Models on Wikipedia-Style Survey Generation: an Evaluation in NLP Concepts''} \cite{Gao2023} 
analiza el impacto significativo de estos modelos, destacando su capacidad para generar 
texto coherente y contextualmente relevante, traducir idiomas y responder preguntas complejas, 
superando con creces a sistemas anteriores. Si bien este avance ha impulsado aplicaciones 
más sofisticadas y accesibles, también ha planteado nuevos desafíos relacionados con 
la eficiencia computacional, el sesgo en los datos y la ética de su uso.


\subsection{Information Extraction (IE, por sus siglas en inglés)}
En la era digital actual, nos enfrentamos a una inmensa cantidad de datos: 
2.5 quintillones de bytes diariamente. Esta explosión de información, proveniente 
de fuentes tan diversas como las redes sociales y la literatura científica, 
ha hecho de la IE un campo crucial dentro del NLP. Se centra en la automatización 
del proceso de identificar y extraer información estructurada a partir de texto 
no estructurado o semiestructurado. Este proceso transforma datos complejos en 
formatos analíticamente útiles, facilitando la búsqueda, visualización y 
el aprovechamiento de la inmensa cantidad de conocimiento latente en el texto, 
con implicaciones significativas en diversas áreas como la inteligencia de negocios \cite{Gursev2023}.

Desde sus inicios en la década de 1950, la IE ha evolucionado gracias a iniciativas 
como las Conferencias de Comprensión de Mensajes, logrando sistemas capaces de 
extraer información con precisión razonable, aunque con margen de mejora en cuanto 
a la complejidad del lenguaje y la inferencia \cite{Grishman1997}. Con el tiempo, se han 
desarrollado una serie de técnicas fundamentales que son esenciales en el campo de la IE \cite{Geek4geeks2024}:

\begin{itemize}
    \item \textbf{Named Entity Recognition}: Esta técnica consiste en identificar 
    entidades (nombres de personas, organizaciones, lugares, fechas) en un texto. 
    Se puede hacer usando reglas predefinidas, métodos estadísticos que analizan 
    la probabi-lidad de que una palabra sea una entidad, o modelos de aprendizaje 
    profundo que aprenden de grandes cantidades de texto.
    \item \textbf{Relation Extraction}: Aquí se busca identificar las conexiones 
    entre las entidades nombradas. Se pueden usar reglas, modelos de aprendizaje 
    automático que aprenden de ejemplos, o modelos que utilizan grandes bases de 
    datos como fuente de entrenamiento. Las redes neuronales también se aplican 
    para clasificar las relaciones entre entidades.
    \item \textbf{Event Extraction}: Esta técnica consiste en identificar eventos 
    que ocurren en un texto, como accidentes o reuniones, y los elementos 
    involucrados (participantes, lugar, tiempo). Esto se puede lograr usando plantillas 
    predefinidas, modelos de aprendizaje automático que aprenden a identificar eventos 
    a partir de ejemplos, o redes neuronales que analizan la estructura del texto para 
    comprender el evento.
    \item \textbf{Correference Resolution}: Se trata de identificar cuando diferentes 
    palabras o frases en un texto se refieren a la misma entidad. Se usan reglas, 
    modelos de aprendizaje automático que analizan características del lenguaje, 
    o redes neuronales que aprenden a seguir las referencias a través del texto.
    \item \textbf{Template Filling}: Esta técnica consiste en extraer información 
    específica de un texto para completar una plantilla predefinida. Se puede lograr 
    utilizando reglas, modelos de aprendizaje automático que clasifiquen la información, 
    o una combinación de ambos.
    \item \textbf{Open Information Extraction}: Esta técnica busca extraer información 
    de una manera más flexible, sin necesidad de definir de antemano las relaciones 
    que se buscan. Se basa en identificar patrones en el texto o mediante modelos 
    estadísticos y de aprendizaje profundo para encontrar relaciones entre las palabras.
\end{itemize}


\subsubsection{Template Filling}\label{section: templateFilling}
Anteriormente se ha mencionado el \textit{Template Filling} como una técnica de extracción 
de información que utiliza una plantilla predefinida para estructurar la información 
extraída de un texto. Esta plantilla actúa como un molde, con espacios o `slots' que 
deben ser rellenados con información específica extraída del texto.

En el libro \textit{``Encyclopedia of Systems Biology''} \cite{Raja2013}, se aborda el 
tema de \textit{``Template Filling, Text Mining''}, donde se resaltan y definen las 
dos componentes fundamentales que nombran esta técnica: la \textit{plantilla} (template) 
y las \textit{reglas de llenado} (fill rules). Una plantilla es un esquema abstracto 
que se define en un dominio de interés, el que a su vez, determina la información 
genérica a extraer y el formato de la salida. Las reglas de llenado, por su parte, 
describen el proceso de extracción de la información, actuando como guía para 
completar la plantilla.

El diseño de una plantilla para extraer información depende del dominio de interés 
y la naturaleza de la tarea. En dependencia del tamaño y la complejidad del 
conjunto de datos a analizar, dos tipos de plantillas son las utilizadas comúnmente: 
las \textit{plantillas planas} y las \textit{plantillas orientadas a objetos}. 
La estructura de las plantillas planas consiste en una serie de espacios 
(que constituyen los atributos), cada uno con cero, una, o más de una posibilidad 
de llenado, que pueden completarse con texto, números, o símbolos de un conjunto definido. 
Las plantillas orientadas a objetos son estructuras de datos que representan información 
compleja mediante la organización de la misma en subplantillas u ``objetos''. 
A diferencia de las plantillas planas, que simplemente listan atributos, las plantillas 
orientadas a objetos permiten modelar escenarios más complejos y relaciones entre 
datos distribuidos en diferentes atributos o subplantillas, facilitando la gestión 
de información con interdependencias.

En el diseño de plantillas para la extracción de información, se identifican tres 
puntos que, según lo expuesto en \textit{``Template Filling, Text Mining''}, son 
necesarios para definir la sintaxis y la semántica de la plantilla, así como para 
el proceso de llenado de la misma:


\begin{itemize}
    \item La \textbf{definición de la plantilla} establece la estructura y el formato 
    para la extracción de datos, especificando las entidades, atributos y su representación. 
    Se centra en la creación de un esquema claro y consistente que guía el proceso, 
    minimizando la ambigüedad y asegurando la uniformidad en la recolección de información. 
    Un diseño preciso de la plantilla es fundamental para la eficiencia y la calidad del proceso de extracción.
    \item Las \textbf{reglas de interpretación} son instrucciones precisas que mapean 
    la información contenida en los documentos fuente a los campos definidos en la plantilla. 
    Estas reglas, que pueden basarse en patrones, ubicación, contexto o combinaciones de éstos, 
    son cruciales para automatizar y estandarizar la extracción, minimizando la intervención humana y 
    maximizando la precisión del proceso.
    \item La \textbf{documentación de casos} (``case law'') consiste en un registro de 
    ejemplos concretos de documentos procesados, incluyendo la información extraída y 
    la resolución de cualquier ambigüedad o conflicto encontrado. Este registro funciona 
    como una base de conocimiento para el perfeccionamiento de las reglas de interpretación, 
    el entrenamiento de sistemas de aprendizaje automático y la evaluación del rendimiento 
    general del proceso de extracción de información.
\end{itemize}

Existen distintas técnicas para abordar las tareas relacionadas con el \textit{Template Filling}. 
Si bien los \textit{métodos basados en reglas} ofrecen control y transparencia, 
su escalabilidad y adaptación a nuevos datos son limitadas. Por otro lado, 
los \textit{enfoques de aprendizaje automático}, como los modelos de lenguaje, 
ofrecen mayor flexibilidad y capacidad de generalización, aunque a costa de una 
menor interpretabilidad. Finalmente, los \textit{métodos híbridos} combinan 
las fortalezas de ambas aproximaciones, aprovechando las reglas para gestionar 
casos específicos y el aprendizaje automático para manejar la variabilidad y 
la generalización, logrando un sistema más robusto y adaptable para diversas tareas.


\subsection{Information Retrieval (IR)}
Durante muchísimos años, la humanidad ha organizado la información para su posterior 
recuperación y uso: los antiguos romanos y griegos registraban información en 
rollos de papiro, algunos de los cuales tenían etiquetas adjuntas que contenían un 
breve resumen para ahorrar tiempo al buscarlos. Los índices o tablas de contenido 
aparecieron por primera vez en los rollos griegos.

El primer representante de repositorios digitales de documentos para búsqueda fue 
el Sistema SMART de Cornell, desarrollado en la década de 1960 \cite{Salton1965}. 
Los primeros sistemas de RI fueron utilizados principalmente por bibliotecarios 
especializados, quienes preparaban un conjunto de consultas o solicitudes de búsqueda, 
las enviaban al sistema todas juntas y luego esperaban a que se procesaran para recibir 
los resultados. Este enfoque no permitía ajustes inmediatos ni respuestas instantáneas, 
algo que hoy es común en cualquier buscador moderno.

El nacimiento de la World Wide Web \cite{Wikipedia} en 1989 y las computadoras modernas 
marcaron un cambio permanente en los conceptos de almacenamiento, acceso y búsqueda 
de colecciones de documentos, haciéndolos accesibles al público en general e indexándolos 
para una recuperación precisa y de gran cobertura.

Este avance en la gestión de información sentó las bases para lo que hoy conocemos 
como IR, un campo clave en la búsqueda y acceso eficiente a datos. 
Según la definición planteada en el libro \textit{``Introduction to Information Retrieval''} \cite{Manning2008}, 
\textit{``la \textbf{recuperación de información} consiste en encontrar material 
(generalmente documentos) de naturaleza no estructurada (usualmente texto) que satisfaga 
una necesidad de información dentro de grandes colecciones 
(generalmente almacenadas en computadoras)''}.

Sin embargo, la IR puede abarcar otros tipos de datos y problemas informáticos 
más allá de lo especificado en la definición antes mencionada. Con el tiempo, 
la recuperación de información ha evolucionado hasta convertirse en la forma 
dominante de acceder a la información, superando incluso a la búsqueda en bases 
de datos tradicionales, donde era necesario proporcionar identificadores específicos.

Esta disciplina no solo se limita a datos estructurados, como en las bases de datos 
relacionales, sino que también abarca datos no estructurados, como los textos, 
que aunque no siempre tienen una estructura evidente, presentan organización 
subyacente como títulos, párrafos y notas al pie. Además, también puede abarcar 
otros tipos de datos y problemas informáticos más allá de lo especificado en la definición 
central mencionada, como la búsqueda en datos semi-estructurados. Un ejemplo de esto 
es cuando se busca un documento que contenga ciertas palabras clave en su título y cuerpo. 
Además de la búsqueda, la recuperación de información incluye el apoyo al usuario en la 
navegación y filtrado de colecciones de documentos, así como el procesamiento de los 
resultados obtenidos. Esto puede incluir tareas como la agrupación de documentos 
basados en su contenido o la clasificación automática según categorías 
predeterminadas \cite{Manning2008}.

Para implementar estos procesos IR, se han desarrollado los Sistemas de Recuperación 
de Información (IRS, por sus siglas en inglés), que son conjuntos de herramientas 
y procesos diseñados para almacenar, organizar, recuperar y presentar información 
de manera eficiente en respuesta a las consultas del usuario. Los IRS están orientados 
a facilitar el acceso a grandes volúmenes de datos, tanto estructurados como no estructurados, 
como documentos, imágenes, videos y otros tipos de contenido digital. Su función principal 
es ayudar a los usuarios a encontrar información relevante dentro de un conjunto de datos, 
basándose en consultas o búsquedas \cite{Ceri2013}.

Las estrategias de recuperación de información asignan una medida de similitud 
entre una consulta y un conjunto de documentos. Estas estrategias se basan en 
la idea de que cuán frecuentes aparecen los mismos términos en ambos.

Sin embargo, para lidiar con las ambigüedades inherentes al lenguaje, como la 
posibilidad de que un mismo concepto sea expresado con diferentes términos, 
algunas estrategias implementan medidas adicionales. Asimismo, un término puede 
tener múltiples significados dependiendo de su contexto, lo que requiere técnicas 
especializadas para garantizar que se interpretan correctamente los conceptos.

En cuanto a su definición formal, una estrategia de recuperación es un algoritmo 
que toma una consulta $Q$ y un conjunto de documentos $D_{1}$, $D_{2}$, ..., $D_{n}$, 
y define una función de ranking, que Grossman asume como el Coeficiente de Similitud 
$SC(Q, D_{i})$ para cada uno de los documentos $1 \leq i \leq n$ \cite{Grossman2004}.

Existen diversas estrategias de recuperación, y la elección del modelo adecuado 
depende de las características del sistema y los requisitos específicos de la consulta. 
A continuación, se presentan algunos de estos enfoques \cite{Baeza1999}.

El \textbf{Modelo Booleano} está basado en la teoría de conjuntos y el álgebra 
de Boole. Las consultas se formulan mediante expresiones booleanas, utilizando 
conectores lógicos como $not$, $and$ y $or$, las cuales tienen una semántica precisa 
y pueden representarse en forma normal disyuntiva. Sin embargo, presenta limitaciones 
importantes. Al basarse en un criterio binario de relevancia, carece de una escala 
de gradación que permita medir la relevancia de manera más precisa, lo que puede 
resultar en la recuperación de muy pocos o demasiados documentos. Además, aunque 
las expresiones booleanas son formalmente claras, a menudo resulta difícil y poco 
intuitivo para los usuarios formular consultas complejas que reflejen sus 
necesidades de información.

A pesar de sus limitaciones, el modelo Booleano sigue siendo relevante en ciertos 
contextos debido a su simplicidad, especialmente para usuarios nuevos en el campo 
de la recuperación de información o en sistemas que no requieren un alto nivel de complejidad.

El \textbf{Modelo de Espacio Vectorial} (VSM, por sus siglas en inglés) propone 
una mejora sobre el modelo booleano al permitir coincidencias parciales mediante 
el uso de pesos no binarios asignados a los términos tanto en las consultas como 
en los documentos. Estos pesos permiten calcular un grado de similitud entre cada 
documento almacenado en el sistema y la consulta del usuario. Al ordenar los documentos 
recuperados en función de este grado de similitud, el modelo logra resultados más precisos 
y relevantes, ajustándose mejor a las necesidades de información del usuario.

En el VSM, tanto los documentos \( d_j \) como las consultas \( q \) se representan 
como vectores \( t \)-dimensionales, donde cada dimensión corresponde a un término 
índice en el sistema. Cada componente del vector de un documento \( \vec{d_j} \) y de 
una consulta \( \vec{q} \) está ponderada por los valores \( w_{i,j} \) y \( w_{i,q} \), 
respectivamente, con \( w_{i,j} \geq 0 \) y \( w_{i,q} \geq 0 \). La similitud entre un 
documento y una consulta se calcula mediante el coseno del ángulo entre sus vectores, 
utilizando la fórmula:

\[
\text{sim}(d_j, q) = \frac{\sum_{i=1}^t w_{i,j} \cdot w_{i,q}}{\sqrt{\sum_{i=1}^t w_{i,j}^2} \cdot \sqrt{\sum_{i=1}^t w_{i,q}^2}}
\]

Aquí, \( |\vec{d_j}| \) y \( |\vec{q}| \) representan las normas de los vectores, 
donde el factor \( |\vec{q}| \) no afecta el ordenamiento de los documentos, 
ya que es constante para todos ellos, mientras que \( |\vec{d_j}| \) proporciona 
una normalización en el espacio de documentos. El valor de similitud \( \text{sim}(d_j, q) \) 
oscila entre 0 y 1, indicando el grado de correlación entre el documento y la consulta. 
Este enfoque no predice directamente si un documento es relevante, sino que clasifica 
los documentos en función de su grado de similitud con la consulta. De este modo, 
un documento puede ser recuperado incluso si solo coincide parcialmente con la consulta. 
Además, es posible establecer un umbral de similitud para filtrar únicamente los documentos 
cuya similitud exceda dicho valor. Para determinar estos valores de similitud, 
es necesario definir cómo se calculan los pesos de los términos índice.

Para calcular estos pesos, se emplea el factor de frecuencia de término (TF), 
que mide la frecuencia con la que un término \( k_i \) aparece en un documento \( d_j \), 
reflejando cuán representativo es dicho término para el contenido del documento. 
A esto se le añade la frecuencia inversa de documento (IDF), que ajusta la relevancia 
de un término en función de cuán común o raro es en la colección completa. 
La frecuencia inversa de documento es especialmente útil para reducir la importancia 
de los términos que aparecen con demasiada frecuencia, ya que no contribuyen 
significativamente a la diferenciación de los documentos. La fórmula para 
calcular el peso TF-IDF es:

\[
\text{TF-IDF}(k_i, d_j) = \frac{f_{i,j}}{\text{max}_i(f_{i,j})} \times \log \left( \frac{N}{n_i} \right),
\]

donde \( f_{i,j} \) es la frecuencia bruta del término \( k_i \) en el documento \( d_j \); 
\( \text{max}_i(f_{i,j}) \) es la frecuencia máxima de cualquier término en el documento 
\( d_j \); \( N \) es el número total de documentos en la colección y \( n_i \) 
es el número de documentos en los que aparece el término \( k_i \).

Gracias a la combinación de los factores TF y IDF, este modelo mejora la precisión 
en la clasificación de documentos. Sin embargo, el VSM presenta una limitación teórica 
importante: asume que los términos dentro de un documento son independientes entre sí, 
lo que puede no reflejar la realidad en documentos donde los términos están relacionados 
o dependen unos de otros. A pesar de esta suposición, el modelo sigue siendo uno de 
los enfoques más utilizados en sistemas de búsqueda debido a su simplicidad y efectividad.

El \textbf{Modelo de Indexación Semántica Latente} (LSI, por sus siglas en inglés) 
es una variante del VSM que aborda problemas como la sinonimia y la polisemia, 
que afectan los modelos clásicos basados en términos índice. LSI utiliza una 
técnica matemática llamada Descomposición en Valores Singulares (SVD), que permite 
representar los datos de una manera más compacta y significativa. Mediante SVD, 
el modelo transforma la matriz de términos y documentos en un espacio de menor 
dimensión, capturando las relaciones semánticas subyacentes entre términos, 
en lugar de simplemente emparejar palabras exactas. Este proceso ayuda a eliminar 
dimensiones que no aportan valor relevante, permitiendo una mejor comprensión y 
recuperación de la información.

El resultado es un modelo más eficiente que captura solo las características más 
relevantes del texto, facilitando el análisis y la recuperación de información pertinente.



\section{Bases de Datos: de los sistemas tradicionales a las soluciones modernas}
A lo largo de la historia, la necesidad de almacenar y organizar datos ha sido fundamental 
para el avance de la civilización, facilitando la transmisión de conocimientos y el 
desarrollo de la ciencia y la tecnología. Desde sus inicios, los seres humanos han 
utilizado sistemas de almacenamiento, como bibliotecas y librerías, que resguardaban 
grandes cantidades de información en libros y documentos.

La transición de los sistemas tradicionales de almacenamiento a soluciones digitales 
surgió como respuesta al crecimiento exponencial de la información. Con la llegada 
de la era digital, se presentó la necesidad no solo de almacenar grandes volúmenes 
de datos, sino de gestionarlos de manera eficiente, escalable y accesible.

El concepto de base de datos tiene sus raíces mucho antes de la llegada de las primeras 
computadoras electrónicas. Vannevar Bush, con su visión sobre la organización de la 
información, propuso estructuras para almacenar datos de manera más flexible, 
sin depender de configuraciones específicas de hardware, lo que influyó en el diseño 
de sistemas de almacenamiento y procesamiento de datos más avanzados \cite{Taylor2001}.

Esta necesidad de un control más eficiente y escalable sobre los datos dio origen 
a la idea de separar la gestión de los datos de la lógica de las aplicaciones, 
lo que resultó en la creación de los Sistemas de Gestión de Bases de Datos (SGBD). 
Estos sistemas actúan como intermediarios entre las aplicaciones y los datos, 
proporcionando herramientas esenciales para definir, crear, consultar, actualizar y 
administrar la información.

En este contexto, Allen Taylor ofrece su propia definición de base de datos:

\begin{quote}
    \textit{“Una base de datos es una colección autodescriptiva de registros integrados. Por autodescriptiva, me refiero a que contiene una descripción de su propia estructura como parte de los datos que almacena. Cuando digo que los registros en una base de datos son integrados, me refiero a que existen relaciones entre ellos que los vinculan, formando un sistema lógico y cohesivo”} \cite{Taylor2001_2}.
\end{quote}

La propuesta de Allen Taylor se amplió y consolidó sobre la base teórica desarrollada por Edgar Codd, 
quien en 1970 introdujo el modelo relacional de bases de datos en su artículo 
titulado \textit{“A Relational Model of Data for Large Shared Data Banks”} \cite{Codd1970}, 
en el que presentó una nueva teoría sobre la organización y gestión de los datos.

Las bases de datos relacionales se basan en un modelo matemático formal que organiza 
los datos y las relaciones entre ellos a través de tablas, donde cada tabla representa 
una relación entre los diferentes elementos almacenados. Cada tabla contiene filas 
(tuplas) y columnas (atributos), donde cada columna representa un atributo específico y 
cada fila almacena los valores correspondientes a esos atributos para una entidad 
particular \cite{Chen2016}. Este modelo se define a través de un esquema que establece 
la estructura de las tablas, los atributos y las relaciones entre ellas. 
En relación con esto, Codd introduce el concepto de consistencia, que se refiere al 
estado que alcanza una base de datos cuando satisface un conjunto de restricciones, 
conocidas como restricciones de integridad, que son utilizadas para garantizar la 
estabilidad y fiabilidad de los datos en la base de datos durante la ejecución de 
operaciones que los modifican.

Los SGBD relacionales se caracterizan por el procesamiento transaccional de los datos, 
es decir, por un conjunto de operaciones que modifican el estado de la base de datos. 
Estos sistemas están construidos sobre los principios ACID, que garantizan la 
consistencia y fiabilidad en la gestión de las transacciones:

\begin{itemize}
    \item \textbf{Atomicity (A)}: Las transacciones son indivisibles; o se completan 
    en su totalidad o no se realizan en absoluto.
    \item \textbf{Consistency (C)}: Después de cada transacción, la base de datos 
    pasa de un estado consistente a otro.
    \item \textbf{Isolation (I)}: Las transacciones no deben interferir entre sí, 
    es decir, el estado intermedio de una transacción no es visible por el resto de 
    transacciones.
    \item \textbf{Durability (D)}: Una vez completada una transacción, sus cambios 
    son permanentes, incluso ante fallos del sistema.
\end{itemize}

A pesar de la robustez que ofrece el modelo relacional, su aplicación universal 
ha comenzado a mostrar limitaciones significativas. Entre los problemas más comunes 
se encuentran: el alto costo de las lecturas, ya que las consultas que involucran 
operaciones de unión (JOIN) entre tablas pueden ser costosas en términos de tiempo 
de ejecución y recursos de cómputo; la sobrecarga de transacciones, que pueden afectar 
el rendimiento si no se requieren para garantizar la integridad de los datos; 
la dificultad para escalar horizontalmente, ya que los SGBD relacionales no están 
diseñados para distribuir datos eficientemente entre varios servidores; y la ineficiencia 
al representar algunos dominios complejos, como los modelos orientados a objetos o 
las redes sociales, que no se ajustan bien al modelo relacional \cite{Camacho}.

Con el objetivo de superar las limitaciones del modelo relacional, surgieron 
las bases de datos NoSQL, que adoptan un enfoque diferente a la gestión de datos. 
En lugar de basarse en las garantías de consistencia estrictas que ofrece ACID, 
los SGBD NoSQL priorizan la disponibilidad del sistema, fundamentándose en los 
principios BASE \cite{AWS}

\begin{itemize}
    \item \textbf{Basically Available (BA)}: Garantiza que el sistema esté disponible 
    para consultas y operaciones de escritura en todo momento, permitiendo la accesibilidad 
    simultánea por parte de los usuarios sin necesidad de esperar a que otros finalicen 
    sus transacciones para actualizar los registros, incluso si no todas las réplicas 
    están al día.
    \item \textbf{Soft state (S)}: Hace referencia a la noción de que los datos pueden 
    tener estados transitorios o temporales que pueden cambiar con el tiempo, incluso 
    sin nuevas entradas. Describe el estado de transición del registro cuando varias 
    aplicaciones lo actualizan en simultáneo. El valor del registro se finaliza solo 
    después de que se hayan completado todas las transacciones. 
    \item \textbf{Eventual consistency (E)}: Asegura que, aunque no haya consistencia 
    inmediata, todas las réplicas del sistema alcanzarán eventualmente un estado consistente.
\end{itemize}

Esto se alinea con las restricciones establecidas por el Teorema CAP \cite{CAP}, 
cuya conjetura fue enunciada por Eric Brewer. Establece que un sistema distribuido 
no puede ofrecer simultáneamente las tres garantías clave de consistencia, disponibilidad 
y tolerancia a particiones. Los SGBD NoSQL, al priorizar la disponibilidad y la tolerancia 
a particiones sobre la consistencia, ofrecen una solución flexible y escalable para el manejo 
de grandes volúmenes de datos, lo que los hace especialmente adecuados para aplicaciones 
modernas, como los sistemas distribuidos y aplicaciones web de alto tráfico.

Los SGBD NoSQL generalmente se diferencian según la forma en que almacenan los datos, 
es decir, el modelo de datos empleado para el almacenamiento. Existen cuatro modelos 
de datos principales implementados en los SGBD NoSQL \cite{Oliveira2021}. La descripción de 
cada uno de estos modelos se presenta a continuación:

Las \textbf{Bases de datos clave-valor} son uno de los sistemas NoSQL más simples, 
donde los datos se almacenan en una tabla sin esquema rígido, con cada fila asociada 
a una clave única y un valor autodescrito, que puede adoptar diversos formatos como 
cadenas, JSON o XML. Los datos se almacenan en pares donde cada clave única está 
asociada a un valor, el cual puede ser un conjunto de datos relacionados, denominado 
agregado. Este agregado se maneja como una unidad completa, lo que significa que al 
recuperar los datos usando una clave, se obtiene todo el conjunto asociado a esa clave. 
La opacidad del agregado implica que el sistema no interpreta ni descompone su contenido; 
lo ve simplemente como un bloque de bits. Debido a esta opacidad, no es posible realizar 
recuperaciones parciales de los datos dentro del agregado, es decir, si solo se necesita 
una parte de la información almacenada, se debe recuperar todo el conjunto de datos 
asociado a la clave, sin posibilidad de acceder a elementos específicos de manera individual.

Las \textbf{Bases de datos columnares} almacenan los datos de manera similar a las 
bases de datos clave-valor. En este modelo, los datos se organizan en una tabla 
sin una estructura rígida, donde cada fila está asociada a una clave única y contiene 
varias familias de columnas. Cada familia de columnas es un conjunto de columnas autodescritas, 
y puede contener solo las columnas relevantes para esa clave, lo que proporciona una gran 
flexibilidad. Además, los datos dentro de una familia de columnas se acceden frecuentemente 
juntos, y debido a que estas columnas contienen sus claves, es posible realizar recuperaciones 
parciales a través de los índices de las columnas.

Las \textbf{Bases de datos orientada a documentos} almacenan los datos en formato de 
documento como XML, JSON o PDF. Los documentos permiten recuperaciones parciales gracias a su 
estructura autodescrita; y pueden contener diferentes atributos para cada clave, permitiendo 
almacenar datos estructurados y semi-estructurados. Es flexible y permite la definición 
de índices sobre los contenidos de los documentos, lo que facilita realizar operaciones 
específicas sobre sus elementos.

Las \textbf{Bases de datos orientadas a grafos} son particularmente útiles para 
representar relaciones complejas entre los datos y son eficientes para identificar patrones. 
Aquí los datos se representan como nodos (entidades) y aristas 
(relaciones entre entidades). En cuanto a las características de los modelos 
de datos presentados hasta ahora, la flexibilidad en la representación de los datos debido 
a la ausencia de un esquema fijo es una de las pocas similitudes entre las bases de 
datos orientadas a grafos y los otros modelos de datos mencionados, ya que tanto 
los vértices como los nodos pueden contener atributos diferentes entre sí.

La elección entre un modelo relacional y uno NoSQL depende de las necesidades específicas 
del sistema y las características de los datos a manejar. Mientras que los SGBD 
relacionales son ideales para aplicaciones que requieren integridad, consistencia y 
transacciones complejas, los sistemas NoSQL resultan más adecuados para manejar 
grandes volúmenes de datos distribuidos, con alta disponibilidad y escalabilidad, 
especialmente en entornos con datos semi-estructurados o no estructurados.
