\chapter*{Introducción}\label{chapter:introduction}
\addcontentsline{toc}{chapter}{Introducción}

Desde la prehistoria, la relación entre la humanidad y las plantas ha sido fundamental 
para la supervivencia y el desarrollo cultural de las civilizaciones. 
Las plantas han servido como fuente de alimento, medicina y recursos básicos, 
un conocimiento que se ha transmitido a través de generaciones, 
incorporándose en muchas culturas, donde su uso se extiende desde el tratamiento de enfermedades 
hasta prácticas espirituales y rituales.

El surgimiento de la agricultura durante el período neolítico revolucionó la historia, 
transformando el modo de vida y la supervivencia humana \cite{Crespo2022}. 
Gracias a la acumulación de conocimientos previos, las sociedades comenzaron a cultivar 
plantas con fines específicos, tanto culinarios como medicinales. 
Esta práctica sentó las bases del conocimiento en antiguas civilizaciones como 
la griega y la romana, donde las plantas no solo formaban parte de la medicina, 
sino de la mitología y la literatura.

En el contexto iberoamericano, las plantas medicinales han adquirido un valor 
estratégico tanto cultural como económico. La herencia de tradiciones ancestrales 
ha contribuido al auge del \enquote{consumo verde} a nivel mundial, 
revitalizando el interés por los remedios naturales y reconociendo así el potencial 
terapéutico de la naturaleza \cite{Ocampo2002}.

La medicina natural y tradicional ha sido, desde hace siglos, un elemento clave 
en la cultura y la identidad del pueblo cubano. Ante cualquier dolencia, es común 
encontrar quien recomiende un remedio casero, cultive su propio huerto medicinal o 
domine el arte de preparar cocimientos con propiedades terapéuticas. Estas prácticas, 
profundamente arraigadas, tienen sus raíces en la mezcla de tradiciones europeas, 
africanas y asiáticas, que confluían en el archipiélago desde la época colonial.

Con el paso de los años, esta riqueza cultural se integró en el sistema de salud cubano, 
ganando un respaldo institucional a partir de eventos clave. 
La Organización Mundial de la Salud (OMS), al finalizar la Conferencia Internacional 
sobre Atención Primaria, celebrada en 1978, emitió su conocida Declaración de Alma Ata, 
la que entre diversas propuestas, realizó un llamado para incorporar las medicinas 
alternativas y terapias tradicionales con eficacia científicamente demostrada, 
a los Sistemas Nacionales de Salud \cite{Ocampo2002}. Sin embargo, fue durante el 
``Período Especial'', tras la caída del campo socialista en 1991, 
cuando el uso organizado de las plantas medicinales adquirió una nueva dimensión en Cuba. 
Frente a la escasez de medicamentos en el país, se llevó a cabo de un Programa de Plantas Medicinales, 
estableciendo bases científicas para su producción y utilización. 
Este programa no solo respondió a una necesidad urgente, sino que también consolidó 
una tradición médica que fusionaba raíces populares y científicas \cite{Lopez2019}.

El conocimiento acumulado a lo largo de la historia no habría sido posible sin la labor 
de destacados investigadores, como Juan Tomás Roig y Mesa, quien en el prólogo 
de su libro \textit{``Plantas medicinales, aromáticas o venenosas de Cuba''} \cite{Roig1945}, 
publicado en 1945, detalla su propósito de documentar y sistematizar el uso de especies 
vegetales con aplicaciones medicinales, culturales y económicas en Cuba. 

Según Roig, su obra pretende ofrecer información \textit{“lo más completa y exacta que sea posible acerca de nuestras plantas medicinales o venenosas”} 
y, además, servir como fuente de consulta para estudiantes y científicos en disciplinas como botánica, farmacia y medicina, 
estimulando el estudio metódico de la flora médica del país. Además, subraya el potencial impacto 
económico de estas plantas al señalar que su estudio podría conducir 
\textit{“a la creación de una industria farmacéutica, que podría proporcionar trabajo a muchos obreros en el campo, y empleo a numerosas personas en los laboratorios y oficinas comerciales”}. 
Esta visión demuestra el compromiso del autor no solo con la preservación del conocimiento, 
sino también con el desarrollo económico y social basado en los recursos naturales.

El esfuerzo de Roig por incluir aspectos como los nombres científicos y vulgares, 
descripciones botánicas y aplicaciones medicinales demuestra su interés por hacer 
el conocimiento accesible tanto para científicos como para el público en general. 
Su obra trasciende como un legado fundamental en la sistematización del uso de plantas 
medicinales en Cuba, contribuyendo al conocimiento científico y práctico, 
e inspirando iniciativas dedicadas a la conservación y estudio de la biodiversidad cubana.

Los esfuerzos históricos de preservación y sistematización del conocimiento sobre 
la flora cubana culminaron en la fundación del Jardín Botánico Nacional de Cuba 
el 24 de marzo de 1968. Esta institución emblemática forma parte de la 
Universidad de La Habana y combina la conservación de la flora con la educación 
ambiental. Se extiende por aproximadamente 600 hectáreas y alberga más de 4,000 
especies vegetales, convirtiéndose en uno de los jardines botánicos más grandes y 
completos del mundo \cite{EcuredJBNC}.

El Jardín Botánico Nacional tiene como misión principal promover el conocimiento 
sobre la flora cubana y tropical, enfatizando la importancia de la conservación 
ambiental. Su enfoque educativo busca involucrar a la población en general, 
ofreciendo un espacio donde se combinan actividades recreativas con la enseñanza 
sobre el medio ambiente, perpetuando los ideales de preservación, educación y 
aprovechamiento sostenible de los recursos naturales \cite{CadenaHabana2022}.

En este contexto, la integración de soluciones tecnológicas que permitan gestionar y 
analizar la información sobre plantas medicinales cubanas se vuelve una necesidad 
estratégica. La sistematización digital del conocimiento no solo facilitaría el acceso 
a datos esenciales para investigadores y estudiantes, sino que también podría impulsar 
nuevas líneas de investigación en biotecnología, farmacología y sostenibilidad ambiental. 
Por ello, aprovechar herramientas modernas como los sistemas de Inteligencia de Negocios 
representa una oportunidad para transformar la gestión de estos datos.

Los sistemas de BI ofrecen funcionalidades que mejoran significativamente la gestión 
y el análisis de datos, destacando el acceso en tiempo real a información actualizada \cite{IbmBI}. 
Esta capacidad no solo permite superar los desafíos asociados a la dispersión y 
el formato físico de los datos, sino que también optimiza de manera sustancial 
la toma de decisiones informadas.

Bajo esta visión, el Jardín Botánico Nacional busca desarrollar una solución computacional 
que integre y gestione la información contenida en la obra de Juan Tomás Roig Mesa. 
Esta herramienta facilitará el acceso y organización de datos sobre las plantas medicinales 
cubanas, contribuyendo al reconocimiento de su importancia cultural y científica. 
Además, permitirá a los investigadores y especialistas disponer de una base estructurada 
para profundizar en el estudio y aplicación de estas plantas en áreas de interés 
económico y social, fortaleciendo el rol del Jardín Botánico como un centro 
de referencia en el ámbito de la medicina natural y la biodiversidad.

La situación descrita permite definir el siguiente \textbf{problema científico}: 
el diseño e implementación de una solución computacional que permita extraer 
la información que ofrece la obra de Juan Tomás Roig y Mesa, en concreto: 
\textit{``Plantas medicinales, aromáticas o venenosas de Cuba''}; y posteriormente 
facilitar el acceso y manipulación de la información científica presente en la misma. 

A partir del problema planteado, se enuncia la siguiente \textbf{hipótesis}: 
la implementación de un sistema computacional para la gestión de la información 
científica basado en la obra de Juan Tomás Roig y Mesa mencionada anteriormente, 
bajo la concepción del desarrollo web y que utilice técnicas de Procesamiento de 
Lenguaje Natural para la manipulación de la información, permitirá extraer el 
conocimiento científico de la obra de Roig y resultará en un sistema de gestión de 
la información que brinde facilidades en cuanto al acceso y manipulación de los datos. 

El \textbf{objetivo general} de este trabajo de diploma es implementar técnicas 
de Procesamiento de Lenguaje Natural para la obtención y estructuración de la información 
contenida en la obra \textit{``Plantas medicinales, aromáticas o venenosas de Cuba''} de 
Juan Tomás Roig y Mesa, y su posterior manejo e integración como base inicial de conocimiento 
en un sitio web nacional para la gestión de la información sobre plantas medicinales cubanas, 
administrado por el Jardín Botánico Nacional de Cuba.

Para alcanzar el cumplimiento del objetivo general del trabajo de diploma, se pueden 
definir un conjunto de objetivos específicos:
\begin{enumerate}
    \item Profundizar en los elementos teórico-conceptuales y prácticos vinculados al 
    procesamiento del lenguaje natural, la minería de textos y el paradigma de 
    bases de datos no relacionales, en especial las bases de datos orientadas 
    a documentos, que posibiliten la fundamentación teórico-metodológica de la propuesta.
    \item Diseñar los modelos de datos y procesos que respondan a los requeri-mientos 
    informacionales en función de los intereses de los especialistas de Botánica, 
    Farmacia, Medicina, Agronomía y Veterinaria.
    \item Diseñar, implementar y evaluar un prototipo de línea de trabajo que permita 
    la digitalización, estructuración y almacenamiento de la información contenida 
    en la obra de Juan Tomás Roig y Mesa respecto a plantas medicinales cubanas.
    \item Diseñar, implementar y evaluar un prototipo de solución computacional que 
    permita la gestión y organización de los datos, además de enriquecer la visualización 
    de los resultados.
\end{enumerate}

A continuación se expone la estructura del documento, que consta de otros tres capítulos 
en los que se detallan las bases, el diseño y la implementación de la solución adoptada.

Capítulo 1 - \textit{``Marco teórico-conceptual"}: Aborda las bases teóricas que fueron 
objeto de estudio para fundamentar los métodos utilizados durante el diseño y puesta en 
práctica de la solución computacional al problema presentado.

Capítulo 2 - \textit{``Concepción y diseño de la solución"}: Expone y caracteriza las 
elecciones en cada parte del proceso de concepción y diseño de la solución computacional, 
desde el punto de vista teórico.

Capítulo 3 -\textit{"Implementación y experimentación"}: Detalla los aspectos técnicos 
de la solución práctica, y se evalúan los resultados.

Posteriormente se presenta un apartado con las conclusiones del trabajo realizado, 
así como la sección de recomendaciones, donde se proponen ideas que pueden ser objetivo 
de investigación para extender la funcionalidad del software, y dotarlo de un mayor 
valor de uso.

Para finalizar, se incluyen las referencias bibliográficas que respaldan la base 
científica de la solución propuesta, así como los anexos.