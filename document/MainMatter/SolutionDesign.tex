\chapter{Concepción y diseño de la solución}\label{chapter:solutionDesign}
En este capítulo se describe la concepción de la solución propuesta para el desarrollo 
del sistema destinado a la gestión de la información científica sobre las 
plantas medicinales cubanas, con base de conociemiento inicial en el libro de Tomás Roig. 
El capítulo está estructurado en tres secciones principales.

En primer lugar, se presenta el contexto en que se desarrolla la solución, 
explicando las motivaciones y necesidades que llevaron a su concepción. 
Posteriormente, se expone la modelación del problema, detallando cómo se definió y 
estructuró la problemática a resolver. 
Finalmente, se describe el diseño de la solución, dividida en dos subproblemas específicos, 
abordando el enfoque adoptado para dar respuesta a cada uno de ellos. 
Este análisis establece las bases conceptuales y técnicas necesarias para la 
posterior implementación y experimentación del sistema.
\newpage



\section{Contexto de la solución}
La iniciativa para desarrollar el presente sistema surge a partir de un diagnóstico 
conjunto realizado entre la Universidad de La Habana y el Jardín Botánico Nacional de Cuba, 
en el contexto de un acuerdo de colaboración científica. 
Este acuerdo tiene como objetivo principal la creación de soluciones que apoyen 
la preservación, organización y difusión del conocimiento botánico en el país, 
un campo que reviste gran importancia tanto para la investigación científica, 
como para la educación y el conociemiento general de las personas.

Durante este diagnóstico, se identificó que uno de los recursos más valiosos 
en el ámbito de la botánica cubana, el libro \textit{``Plantas medicinales, aromáticas o venenosas de Cuba''} 
de Tomás Roig y Mesa, enfrentaba múltiples desafíos relacionados con su accesibilidad y 
aprovechamiento. Este libro, publicado originalmente en 1945, constituye una obra 
de referencia fundamental que recopila una vasta cantidad de información científica 
sobre la flora medicinal de Cuba, incluyendo descripciones botánicas, 
usos terapéuticos y distribución geográfica de las plantas documentadas. 
Sin embargo, a pesar de su relevancia, el acceso a esta información sigue siendo 
limitado debido a varios factores:

\begin{itemize}
    \item \textbf{Formato físico predominantemente tradicional}: Aunque existen versiones 
    digitales del libro, estas no cuentan con un diseño modular que facilite su consulta 
    o análisis de la información. Esto reduce significativamente su usabilidad en contextos 
    modernos donde predominan las herramientas tecnológicas.
    \item \textbf{Pérdida potencial del conocimiento}: El envejecimiento de los ejemplares físicos 
    y la falta de iniciativas de conservación digital de alta calidad ponen en riesgo 
    la preservación de este importante recurso.
    \item \textbf{Falta de integración en sistemas modernos de información}: Los datos 
    contenidos en el libro no están organizados de manera que puedan ser utilizados 
    en aplicaciones automatizadas, análisis de datos o sistemas de consulta avanzada.
\end{itemize}

A partir de esta realidad, el Jardín Botánico Nacional planteó la necesidad de desarrollar 
un sistema que no solo permitiera la digitalización de esta información, 
sino que también la estructurara en un formato accesible y flexible, 
capaz de responder a las demandas de diferentes tipos de usuarios. 
Este sistema debía estar alineado con el interés institucional de promover la conservación 
del patrimonio científico y natural de Cuba, a la vez que facilitara su divulgación 
a nivel nacional.

La Universidad de La Habana, como institución de referencia en la formación de 
profesionales en ciencias y tecnología, asumió el reto de apoyar esta iniciativa 
mediante el desarrollo de una solución tecnológica que integre técnicas de recuperación 
de información y bases de datos científicas. Este proyecto, en particular, 
representa un esfuerzo no solo por preservar los recursos botánicos, 
sino también por sentar las bases para la creación de sistemas similares que puedan 
aplicarse a otros ámbitos del conocimiento.





\section{Modelación del Problema}




\section{Diseño de la solución}
\subsection{Extracción de la información}
\subsection{Sistema de gestión y visualización}