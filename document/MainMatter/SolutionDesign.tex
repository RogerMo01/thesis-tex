\chapter{Concepción y diseño de la solución}\label{chapter:solutionDesign}
En este capítulo se describe la concepción de la solución propuesta para el desarrollo 
del sistema destinado a la gestión de la información científica sobre las 
plantas medicinales cubanas, con base de conociemiento inicial en el libro de Tomás Roig. 
El capítulo está estructurado en tres secciones principales.

En primer lugar, se presenta el contexto en que se desarrolla la solución, 
explicando las motivaciones y necesidades que llevaron a su concepción. 
Posteriormente, se expone la modelación del problema, detallando cómo se definió y 
estructuró la problemática a resolver. 
Finalmente, se describe el diseño de la solución, dividida en dos subproblemas específicos, 
abordando el enfoque adoptado para dar respuesta a cada uno de ellos. 
Este análisis establece las bases conceptuales y técnicas necesarias para la 
posterior implementación y experimentación del sistema.
\newpage



\section{Contexto del problema}
La iniciativa para desarrollar el presente sistema surge a partir de un diagnóstico 
conjunto realizado entre la Universidad de La Habana y el Jardín Botánico Nacional de Cuba, 
en el contexto de un acuerdo de colaboración científica. 
Este acuerdo tiene como objetivo principal la creación de soluciones que apoyen 
la preservación, organización y difusión del conocimiento botánico en el país, 
un campo que reviste gran importancia tanto para la investigación científica, 
como para la educación y el conociemiento general de las personas.

Durante este diagnóstico, se identificó que uno de los recursos más valiosos 
en el ámbito de la botánica cubana, el libro \textit{``Plantas medicinales, aromáticas o venenosas de Cuba''} 
de Tomás Roig y Mesa, enfrentaba múltiples desafíos relacionados con su accesibilidad y 
aprovechamiento. Este libro, publicado originalmente en 1945, constituye una obra 
de referencia fundamental que recopila una vasta cantidad de información científica 
sobre la flora medicinal de Cuba, incluyendo descripciones botánicas, 
usos terapéuticos y distribución geográfica de las plantas documentadas. 
Sin embargo, a pesar de su relevancia, el acceso a esta información sigue siendo 
limitado debido a varios factores:

\begin{itemize}
    \item \textbf{Formato físico predominantemente tradicional}: Aunque existen versiones 
    digitales del libro, estas no cuentan con un diseño modular que facilite su consulta 
    o análisis de la información. Esto reduce significativamente su usabilidad en contextos 
    modernos donde predominan las herramientas tecnológicas.
    \item \textbf{Pérdida potencial del conocimiento}: El envejecimiento de los ejemplares físicos 
    y la falta de iniciativas de conservación digital de alta calidad ponen en riesgo 
    la preservación de este importante recurso.
    \item \textbf{Falta de integración en sistemas modernos de información}: Los datos 
    contenidos en el libro no están organizados de manera que puedan ser utilizados 
    en aplicaciones automatizadas, análisis de datos o sistemas de consulta avanzada.
\end{itemize}

A partir de esta realidad, el Jardín Botánico Nacional planteó la necesidad de desarrollar 
un sistema que no solo permitiera la digitalización de esta información, 
sino que también la estructurara en un formato accesible y flexible, 
capaz de responder a las demandas de diferentes tipos de usuarios. 
Este sistema debía estar alineado con el interés institucional de promover la conservación 
del patrimonio científico y natural de Cuba, a la vez que facilitara su divulgación 
a nivel nacional.

La Universidad de La Habana, como institución de referencia en la formación de 
profesionales en ciencias y tecnología, asumió el reto de apoyar esta iniciativa 
mediante el desarrollo de una solución tecnológica que integre técnicas de recuperación 
de información y bases de datos científicas. Este proyecto, en particular, 
representa un esfuerzo no solo por preservar los recursos botánicos, 
sino también por sentar las bases para la creación de sistemas similares que puedan 
aplicarse a otros ámbitos del conocimiento.





\section{Modelación del Problema}
Luego de un análisis exhaustivo de los objetivos del sistema y las necesidades 
identificadas, se han definido los siguientes requerimientos funcionales. 
Estos buscan garantizar una fiel representación de la información, así como 
lograr una interacción fluida y efectiva, 
satisfaciendo las expectativas de los distintos tipos de usuarios a los
que está destinado el producto final.

\begin{itemize}
    \item Presentar de forma estructurada y comprensible toda la información contenida 
    en las monografías del libro de Tomás Roig. Esto incluye las diferentes secciones, 
    como nombres científicos, hábitat, propiedades medicinales y composición química. 
    La representación visual debe facilitar el acceso y la interpretación de los datos, 
    con un diseño que priorice la claridad.
    \item Proveer un mecanismo avanzado de búsqueda, que permita consultar la información
    de las plantas almacenadas en el sistema, no solo mediante el nombre de las mismas, 
    sino mediante el contexto que ofrece su monografía.
    \item Incluir un módulo administrativo que permita a un usuario administrador gestionar
    la información almacenada. Esto incluye la creación de nuevas monografías de plantas, 
    la edición de información existente para corregir o actualizar datos y la eliminación 
    de registros que ya no sean relevantes o que presenten inconsistencias.
    \item Ofrecer la visualización de otras secciones relevantes del libro, que enriquezcan el
    acceso a la información.
\end{itemize}

Para comprender mejor las interacciones de los usuarios y las funcionalidades del sistema, 
se ha elaborado un diagrama de casos de uso. Este diagrama representa de manera gráfica 
las principales acciones que los diferentes tipos de usuarios pueden realizar dentro del sistema. 
Su objetivo principal es proporcionar una visión clara y estructurada de los requisitos funcionales, 
destacando los roles de los usuarios y sus respectivos casos de uso. Además, facilita la 
identificación de los límites del sistema, asegurando que las interacciones previstas cubran 
todas las necesidades y expectativas planteadas durante la modelación del problema.

En la figura \ref{fig:use-cases}, se presenta el diagrama de casos de uso correspondiente a la modelación 
propuesta del problema.

\begin{figure}[ht!]
    \centering
    \includegraphics[width=1\textwidth]{Images/use-cases.png}
    \caption{Diagrama de casos de uso}
    \label{fig:use-cases}
\end{figure}

Es posible identificar dos subproblemas principales dentro del contexto del problema anteriormente descrito. 
Estos subproblemas están interrelacionados y son fundamentales para garantizar que el sistema cumpla 
con los objetivos establecidos:

\begin{itemize}
    \item \textbf{Problema de la extracción de la información}: Este subproblema se refiere al proceso 
    de extraer, estructurar y almacenar de manera eficiente la información contenida en el libro de Tomás Roig, 
    de forma que estos datos puedan ser consumidos por un software computacional.
    \item \textbf{Problema del sistema de gestión y visualización de la información}: Una vez 
    extraída y estructurada la información, surge el desafío de diseñar e implementar un sistema que permita 
    gestionar y visualizar eficientemente los datos.
\end{itemize}




\newpage 
\section{Diseño de la solución}
Esta sección tiene como objetivo presentar las estrategias y propuestas de diseño abstracto desarrolladas 
para abordar los subproblemas identificados anteriormente. 
Estos subproblemas, requieren soluciones específicas que garanticen tanto 
la fidelidad de los datos extraídos como su presentación efectiva a los usuarios finales.

En esta sección, se describirán los enfoques conceptuales diseñados para resolver cada 
uno de los subproblemas, teniendo en cuenta los requerimientos funcionales previamente establecidos. 
Se analizarán las características principales de cada solución, incluyendo sus componentes clave y cómo 
estos se integran para formar un sistema coherente y eficiente. Este análisis establecerá las bases 
para la implementación detallada del sistema.


\subsection{Extracción de la información}
Una solución al problema planteado debe partir de un análisis exhaustivo del corpus sobre el cual se 
realizará la extracción de información. En este sentido, tras examinar detalladamente el libro 
\textit{``Plantas medicinales, aromáticas o venenosas de Cuba''}, se identificaron una serie de observaciones 
preliminares. Estas observaciones constituyen la base para tomar decisiones informadas respecto 
a las estrategias de extracción de información que sean adecuadas y acordes con el estado del arte.

Las observaciones identificadas son las siguientes:
\begin{enumerate}
    \item La obra está dividida en dos tomos, por lo que es deseable que la solución adoptada sea 
    lo más general posible, permitiendo su aplicación efectiva en ambos tomos.
    \item Ambos tomos se encuentran en el formato: \textit{Portable Document Format} (PDF)
    \item Aunque pertenecen a la misma editorial, existen diferencias significativas 
    en cuanto a la maquetación y el diseño digital entre ambos tomos.
    \item Las monografías constituyen la sección fundamental del libro.
    \item El tomo 1 contiene las monografías de las plantas cuyos nombres inician con las letras de 
    la `A' a la `K', mientras que el tomo 2 abarca aquellas cuyas iniciales están entre la `L' y la `Z'.
    \item En las monografías se pueden identificar secciones principales que contienen cierta información 
    sobre las plantas. Estas secciones mantienen un orden fijo, aunque no siempre están 
    presentes todas en cada monografía. Las secciones identificadas son:
    \begin{itemize}
        \item Nombre con que se conoce la plantas.
        \item Nombre científico.
        \item Sinónimos.
        \item Otros nombres vulgares asociados.
        \item Hábitat y distribución geográfica.
        \item Descripción botánica.
        \item Composición química.
        \item Partes empleadas.
        \item Propiedades medicinales.
        \item Aplicaciones.
        \item Cultivo.
        \item Referencias bibliográficas.
    \end{itemize}
    \item Algunas monografías incluyen imágenes de baja calidad de las plantas, 
    acompañadas de un pie de foto que identifica el nombre de la especie.
    \item El formato de presentación del texto no es uniforme, lo que responde a decisiones editoriales y 
    de diseño. Por ejemplo, las monografías están dispuestas en una sola columna para facilitar la lectura continua, 
    mientras que otras secciones como la dedicada a la agrupación de plantas según sus aplicaciones 
    están organizadas en tres columnas para optimizar el uso del espacio al listar múltiples nombres.
\end{enumerate}

Estas características del corpus son consideradas para garantizar que la solución propuesta sea capaz de 
abordar los retos específicos que plantea la extracción de información en un contexto tan heterogéneo.

Es factible realizar la extracción del texto contenido en los documentos en formato \textit{PDF} mediante el uso 
de lenguajes de programación modernos, apoyándose en bibliotecas especializadas para la manipulación y 
el procesamiento de este tipo de archivos. No obstante, debe considerarse que el texto presenta 
características de maquetación no uniformes a lo largo de la obra, lo que podría requerir un manejo 
cuidadoso de las estructuras y formatos para asegurar una extracción precisa y completa.

En la sección \ref{section: nlp} se describe el problema general de la IE, identificándola como una 
de las tareas más comunes y relevantes en el ámbito del Procesamiento del NLP. La IE
se enfoca en identificar, estructurar y representar conocimiento relevante a partir de 
textos no estructurados o semiestructurados.

Dadas las características del corpus objeto de estudio y su estructura textual, la técnica seleccionada 
para llevar a cabo el proceso de extracción de información es la denominada \textit{Template Filling} abordada en la sección \ref{section: templateFilling}. 
Esta técnica permite extraer información específica mediante la identificación de patrones 
predefinidos y su mapeo en plantillas estructuradas. La elección de esta metodología 
responde a varios factores:
\begin{enumerate}
    \item La necesidad de obtener los datos en un formato estructurado que facilite su posterior uso
    en sistemas computacionales.
    \item La importancia de preservar la fidelidad de las palabras y expresiones originales del autor 
    para garantizar una representación precisa del contenido de la obra.
    \item La adecuación de esta técnica para procesar textos con una organización semiuniforme, 
    como las monografías presentes en la obra, permitiendo capturar información clave como 
    nombres científicos, descripciones botánicas, propiedades y aplicaciones.
\end{enumerate}

Para mayor conveniencia, se optará por realizar la extracción de información del libro de manera segmentada, 
abordando cada sección de forma independiente. Este enfoque permite aplicar el proceso de \textit{Template Filling} 
a cada sección por separado, lo cual simplifica significativamente los algoritmos necesarios para la extracción.



\subsubsection{Template Filling en monografías}
Para la extracción de información de las monografías, se optará por un enfoque basado en reglas, 
considerando las características semiestructuradas de los datos presentes en esta sección del libro. 
El diseño de la plantilla requerirá abordar tres aspectos fundamentales: la definición 
de la estructura de la plantilla, la especificación de las reglas de interpretación 
y la documentación de casos. Sin embargo, este último punto no será desarrollado, 
dado que su utilidad sería limitada en ausencia de un enfoque basado en aprendizaje automático.

En una etapa inicial, es posible extraer la información correspondiente a cada monografía 
de manera básica. Esto implica definir una plantilla inicial que incluya los nombres de 
todas las plantas mencionadas en el libro, asociando a cada nombre el texto plano que 
representa la información respectiva. La regla de interpretación empleada en este paso se 
encargará de identificar el inicio de cada monografía, utilizando como criterio el nombre de 
la planta, que se encuentra destacado con un tamaño de fuente significativo en el texto.

A partir de esta extracción inicial, se procederá a estructurar la información de cada monografía 
basándose en su contenido. En la Figura \ref{fig:template-monograph}, se presenta la definición de la plantilla utilizada 
para este propósito, en la que se incluye el nombre de cada atributo, junto con el tipo de dato del mismo.

\begin{figure}[ht!]
    \centering
    \includegraphics[width=1\textwidth]{Images/templates.png}
    \caption{Plantillas de monografía y nombre científico}
    \label{fig:template-monograph}
\end{figure}

Cada atributo de la plantilla corresponde a una sección identificable dentro del contenido 
de una monografía. Por ejemplo, 
\texttt{Sc} representa el nombre científico, 
\texttt{Sy} los sinónimos,
\texttt{Vul} los nombres vulgares asociados, 
\texttt{Hab} el hábitat y distribución geográfica, 
\texttt{Des} la descripción botánica, 
\texttt{Cmp} la composición química, 
\texttt{Use} las partes empleadas, 
\texttt{Pro} las propiedades medicinales, 
\texttt{App} las aplicaciones, 
\texttt{Cul} el cultivo y 
\texttt{Bib} las referencias bibliográficas.

Como se observa en la Figura \ref{fig:template-monograph}, este diseño emplea una plantilla híbrida 
que combina características de plantillas planas y plantillas orientadas a objetos, 
permitiendo que los atributos puedan almacenar tanto datos primitivos como subplantillas, 
lo que facilita una representación más estructurada y flexible de la información.

Definamos entonces las reglas de interpretación para la plantilla de las monografías:
\begin{enumerate}
    \item Las secciones dentro de una monografía siempre aparecen en el mismo orden, y no necesariamente aparecen todas en una monografía.
    \item El nombre científico siempre aparece en la primera línea inmediatamente después del título de la monografía.
    \item Los sinónimos están precedidos por la cadena de texto \texttt{``SINÓNIMOS:''} y se encuentran separados entre sí por comas (\texttt{,}).
    \item Los otros nombres vulgares están precedidos por la cadena de texto \texttt{``OTROS NOMBRES VULGARES:''}. Los nombres correspondientes a un mismo territorio están separados por comas (\texttt{,}), mientras que los nombres entre territorios están separados por punto y coma (\texttt{;}).
    \item El texto correspondiente al hábitat y distribución está precedido por la cadena de texto \texttt{``HÁBITAT Y DISTRIBUCIÓN:''}.
    \item El texto correspondiente a la descripción botánica está precedido por la cadena de texto \texttt{``DESCRIPCIÓN BOTÁNICA:''}.
    \item El texto correspondiente a la composición química está precedido por la cadena de texto \texttt{``COMPOSICIÓN:''}.
    \item El texto correspondiente a las partes empleadas está precedido por la cadena de texto \texttt{``PARTES EMPLEADAS:''}.
    \item El texto correspondiente a las propiedades de la planta está precedido por la cadena de texto \texttt{``PROPIEDADES:''}.
    \item El texto correspondiente a las aplicaciones está precedido por la cadena de texto \texttt{``APLICACIONES:''}.
    \item El texto correspondiente al cultivo de la planta está precedido por la cadena de texto \texttt{``CULTIVO:''}.
    \item El texto correspondiente a las referencias bibliográficas está precedido por la cadena de texto \texttt{``BIBLIOGRAFÍA''}, y cada bibliografía termina en el año correspondiente a la misma.
\end{enumerate}

En cuanto a los nombres científicos, se pueden definir las siguientes reglas en función de la plantilla:
\begin{enumerate}
    \item Las partes que componen un nombre científico siempre siguen un orden específico, aunque no necesariamente todas deben estar presentes.
    \item Las primeras dos palabras corresponden al género y la especie, respectivamente.
    \item La autoridad de la planta siempre aparece inmediatamente después del género y la especie.
    \item La variedad siempre está precedida por la cadena de texto \texttt{``var.''}.
    \item La subespecie siempre está precedida por la cadena de texto \texttt{``subsp.''}.
    \item La forma siempre está precedida por la cadena de texto \texttt{``f.''}.
    \item La familia siempre está precedida por la cadena de texto \texttt{``Fam.''}.
    \item La subfamilia siempre está precedida por la cadena de texto \texttt{``Subfam.''}.
\end{enumerate}

Con la definición de estas plantillas y reglas de interpretación, se logra un diseño adecuado para 
la extracción de la información de las monografías, asegurando que los datos sean identificados y 
estructurados correctamente de acuerdo con su formato original.


\subsubsection{Template Filling en agrupación de plantas por aplicaciones}


\subsection{Sistema de gestión y visualización}