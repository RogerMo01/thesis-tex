\begin{resumen}
	Este trabajo de diploma presenta \textbf{BotaniQ}, una plataforma web diseñada 
	para la consulta y gestión de información sobre plantas cubanas, 
	basada en la obra \textit{``Plantas medicinales, aromáticas o venenosas de Cuba''} 
	de Juan Tomás Roig y Mesa. El objetivo principal ha sido desarrollar una 
	solución computacional que facilitara el acceso, la organización y la 
	manipulación de la valiosa información contenida en dicho libro. 
	Para ello, se implementaron técnicas de Procesamiento del Lenguaje Natural, 
	específicamente \textit{template filling}, para extraer y estructurar la 
	información de las monografías y la sección de agrupación de plantas por 
	aplicaciones. Esta información se almacena y gestiona en una base de datos 
	relacional PostgreSQL, aprovechando las flexibilidades que este sistema ofrece, 
	a pesar de ser SQL, para modelar las particularidades de la información extraída. 
	La plataforma ofrece una interfaz intuitiva y un potente sistema de búsqueda contextual, 
	basado en el Modelo de Espacio Vectorial, que permite a los usuarios encontrar 
	información relevante incluso con términos de búsqueda aproximados. BotaniQ 
	incluye también un módulo administrativo para la gestión y actualización de 
	los datos, garantizando la integridad y consistencia de la información. 
	Las pruebas experimentales realizadas confirman la funcionalidad y eficiencia 
	del sistema, demostrando su potencial como una herramienta valiosa para la 
	difusión y conservación del conocimiento sobre la flora medicinal cubana. 
	Finalmente, se proponen mejoras para optimizar el rendimiento y la experiencia 
	de usuario. Este trabajo sienta las bases para futuras investigaciones y 
	el desarrollo de plataformas similares aplicables a otros recursos bibliográficos 
	del campo de la botánica.
\end{resumen}

\begin{abstract}
	\foreignlanguage{english}{
		This diploma thesis presents \textbf{BotaniQ}, a web platform designed for the 
		consultation and management of information on Cuban plants, based on the book 
		\textit{``Plantas medicinales, aromáticas o venenosas de Cuba''} by Juan Tomás Roig y Mesa. 
		The main objective has been to develop a computational solution that facilitates access to, 
		organization of, and manipulation of the valuable information contained in this book. 
		To achieve this, Natural Language Processing techniques, specifically \textit{template filling}, 
		were implemented to extract and structure the information from the monographs and 
		the section grouping plants by applications. This information is stored and managed 
		in a relational PostgreSQL database, leveraging the flexibility that this system offers 
		despite being SQL, to model the particularities of the extracted data. The platform 
		provides an intuitive interface and a powerful contextual search system based on the 
		Vector Space Model, allowing users to find relevant information even with approximate 
		search terms. BotaniQ also includes an administrative module for data management and 
		updates, ensuring the integrity and consistency of the information. The experimental 
		tests conducted confirm the system’s functionality and efficiency, demonstrating its 
		potential as a valuable tool for the dissemination and preservation of knowledge about 
		Cuban medicinal flora. Finally, improvements are proposed to optimize performance and 
		user experience. This work lays the foundation for future research and the development 
		of similar platforms applicable to other bibliographic resources in the field of botany.
	}
\end{abstract}