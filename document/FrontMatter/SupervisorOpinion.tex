\setstretch{1.1}
\begin{opinion}
    Las plantas medicinales representan un recurso invaluable para la salud, ya que contienen compuestos con propiedades terapéuticas utilizadas en la prevención y tratamiento de diversas enfermedades. 
    Su uso, basado en el conocimiento tradicional y la investigación científica, las convierte en una alternativa natural y complementaria a la medicina convencional.
    El conocimiento de la medicina tradicional se ha transmitido de generación en generación, preservando saberes antiguos sobre el uso de plantas y remedios naturales.
    Para garantizar su continuidad, es fundamental documentar y difundir esta información, integrando el conocimiento tradicional dentro del marco de las nuevas tecnologías.
    
    El presente trabajo contribuye a la preservación del conocimiento de la medicina tradicional
    cubana con la digitalización de la obra
    \textit{``Plantas medicinales, aromáticas o venenosas de Cuba''} de Juan Tomás Roig y Mesa
    mediante el empleo de técnicas de procesamiento de lenguaje natural.
    Además, se proporciona a los expertos una plataforma digital, basada en métodos
    de sistemas de recuperación de información, para la consulta de la versión
    digitalizada del libro con vistas al estudio y difusión del conocimiento
    de plantas medicinales cubanas. Es de resaltar el aporte que representa la combinación
    de técnicas basadas en reglas con grandes modelos de lenguaje para asegurar un
    correcto procesamiento del libro.

    Durante el desenvolvimiento del trabajo los diplomantes realizaron estudios
    teóricos en el marco de las bases de datos, tanto relacionales como no relacionales,
    el procesamiento del lenguaje natural, los sistemas de recuperación de información
    y obtuvieron un entendimiento fundamental, desde un punto de vista informacional, sobre
    las plantas y sus usos medicinales. Asimismo, 
    se adentraron en la concepción e implementación de una metodología para
    la digitalización y estructuración de la información contenida en la obra
    de Tomás Roig. De igual modo, concibieron una solución computacional 
    orientada a la web bajo la concepción de la recuperación de información
    para consultar la información obtenida. Al respecto del desarrollo computacional, los estudiantes asimilaron
    e integraron de manera independiente las tecnologías y las herramientas
    vinculadas con el procesamiento de documentos, grandes modelos de lenguaje, 
    las bases de datos relacionales, el desarrollo de
    ambientes web y las técnicas para la recuperación de información. La comprobación de la validez del prototipo que responde a la concepción de una primera aproximación 
    en el marco de
    una solución de inteligencia de negocios, así como la valoración de las fortalezas y las debilidades en términos de la aplicación y la evolución ulterior
    exitosa, se abordan no solo como cierre de la presente tesis sino como puerta
    a la consecución de nuevos logros en el quehacer para la preserservación,
    digitalización y promoción, no solo de la medicina tradicional cubana, sino
    del conocimiento científico cubano en general. 

    Los diplomantes Roger Moreno Gutiérrez y Claudia Alvarez Martínez han desplegado una labor
encomiable como estudiantes de pregrado. Su trabajo se ha caracterizado en todo momento por la dedicación y la independencia, así como por
la profundidad en el análisis crítico y la creatividad en la integración de conocimientos teóricos y prácticos para la obtención de resultados altamente
satisfactorios. La memoria escrita, en nuestra opinión, cuenta con la calidad
y el grado de terminación requeridos, así como con una correcta redacción
y organización de los contenidos, demostrando no solo dominio del procesamiento del lenguaje natural,
de la tecnología SQL y los sistemas de recuperación de información,
sino también aptitud para el desarrollo de soluciones computacionales novedosas. 
Finalmente, podemos afirmar que los estudiantes han cumplido con
los objetivos trazados y han sentado un valioso precedente para continuar
acometiendo una línea de investigación de suma importancia dirigida a preservar y
promover el conocimiento científico cubano.
Por todo lo anteriormente expresado, consideramos que los diplomantes
Roger Moreno Gutiérrez y Claudia Alvarez Martínez  han adquirido con creces todos los conocimientos y
las habilidades necesarias para  ser acreedores de los títulos de Licenciado
en Ciencia de la Computación y Licenciada en Ciencia de la Computación respectivamente, y proponemos al tribunal que se les otorgue
la calificación máxima de 5 puntos (Excelente).

\vspace{20mm}
\centering
Lucina García Hernández \hspace{10mm}  Victor Manuel Cardentey Fundora

\begin{center}
    \textit{Tutores del presente trabajo}
\end{center}

\end{opinion}
\setstretch{1.2}
